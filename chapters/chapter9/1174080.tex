\section{1174080 - Handi Hermawan}

\subsection{Teori}
\begin{enumerate}

        \item Jelaskan dengan ilustrasi gambar sendiri apa perbedaan antara vanilla GAN dan cGAN.

Vanilla GAN merupakan tipe GAN paling sederhana. Yaitu Generator dan Diskriminator adalah perceptron multi-layer sederhana. Dalam vanilla GAN, algoritma ini sangat sederhana, ia mencoba untuk mengoptimalkan persamaan matematika menggunakan keturunan gradien stokastik.Dan CGAN (Conditional GAN), CGAN dapat digambarkan sebagai metode pembelajaran yang mendalam di mana beberapa parameter bersyarat ditempatkan. Di CGAN, parameter tambahan ‘y’ ditambahkan ke Generator untuk menghasilkan data yang sesuai. Label juga dimasukkan ke dalam input ke Diskriminator agar Diskriminator membantu membedakan data nyata dari data yang dihasilkan palsu. 

	\begin{figure}[H]
            	\includegraphics[width=4cm]{figures/1174080/9/teori1.PNG}
				\includegraphics[width=4cm]{figures/1174080/9/teori1,1.PNG}
           	\centering
           	\caption{soal 1}
        	\end{figure}

        \item Jelaskan dengan ilustrasi gambar sendiri arsitektur dari Age-cGAN.

Age cGAN ialah dengan mengkondisikan model pada informasi tambahan dimungkinkan untuk mengarahkan proses pembuatan data, terdiri dari tiga tahap:
Pelatihan bersyarat GAN: Pada tahap ini, kami melatih jaringan generator dan jaringan diskriminator.
Perkiraan vektor laten awal: Pada tahap ini, kami melatih jaringan pembuat enkode.
Optimalisasi vektor laten: Pada tahap ini, kami mengoptimalkan encoder dan jaringan generator.
	\begin{figure}[H]
		\includegraphics[width=4cm]{figures/1174080/9/teori2,10.PNG}
            	\centering
           	\caption{soal 2}
       	 \end{figure}

        \item Jelaskan dengan ilustrasi gambar sendiri arsitektur encoder network dari AgecGAN.

Arsitektur encoder biasanya digunakan untuk memodelkan struktur manifold dan membalikkan encoder untuk memproses data.

	\begin{figure}[H]
		\includegraphics[width=4cm]{figures/1174080/9/teori3.PNG}
            	\centering
           	\caption{soal 3}
       	 \end{figure}

        \item Jelaskan dengan ilustrasi gambar sendiri arsitektur generator network dari AgecGAN.

Arsitektur generator adalah sebuah array yang digunakan secara random, yang disebut seed. dari data seed tersebut, generator akan merubahnya menjadi sebuah gambar yang ukuran 28 x 28 dengan menggunakan Convolutional Neural Network.

	\begin{figure}[H]
		\includegraphics[width=4cm]{figures/1174080/9/teori4.PNG}
            	\centering
           	\caption{soal 4}
       	 \end{figure}


        \item Jelaskan dengan ilustrasi gambar sendiri arsitektur discriminator network dari Age-cGAN.

Arsitektur diskriminator adalah CNN yang dapat menerima input gambar yang berukuran 28,28 serta menghasilkan angka biner yang menyatakan apakah data yang diinputkan merupakan dataset asli atau gambar dataset palsu.

	\begin{figure}[H]
		\includegraphics[width=4cm]{figures/1174080/9/teori5.PNG}
            	\centering
           	\caption{soal 5}
       	 \end{figure}


        \item Jelaskan dengan ilustrasi gambar apa itu pretrained Inception-ResNet-2 Model.

Pre-Trained Network atau Transfer Learning merupakan suatu metode penyelesaian yang memanfaatkan model yang sudah dilatih terhadap suatu dataset untuk menyelesaikan masalah dengan cara menggunakan sebagai starting point, memodifikasi dan mengupdate parameternya, sehingga sesuai dengan dataset yang baru.

	\begin{figure}[H]
		\includegraphics[width=4cm]{figures/1174080/9/teori6.PNG}
            	\centering
           	\caption{soal 6}
       	 \end{figure}

        \item Jelaskan dengan ilustrasi gambar sendiri arsitektur Face recognition network Age-cGAN.

Face Recognition merupakan salah satu sistem yang mengimplementasi Deep Learning yang dapat mengenali wajah secara fisik dari gambar digital atau video frame, sehingga identitas orang tersebut dapat dipertahankan melalui rekonstruksi.

	\begin{figure}[H]
		\includegraphics[width=4cm]{figures/1174080/9/teori7.PNG}
            	\centering
           	 \caption{soal 7}
       	 \end{figure}

        \item Sebutkan dan jelaskan serta di sertai contoh-contoh tahapan dari Age-cGAN.

Pada dari Age-cGan ni terdapat 2 tahapan dengan generator dan diskriminator. dimana untuk tahap generator sendiri membutuhkan vektor laten 100 serta menghasilkan gambar yang realistis dari dimensinya. sedangkan tahap diskriminator itu tahapan dimana memprediksi gambar yang diberikan nyata atau palsu.

	\begin{figure}[H]
		\includegraphics[width=4cm]{figures/1174080/9/teori8.PNG}
            	\centering
           	 \caption{soal 8}
       	 \end{figure}

        \item Berikan contoh perhitungan fungsi training objektif.

Objektif Trainning ialah untuk meminimalkan loss function sebagai log likelihood function yang diberikan pada persamaan dimana D melambangkan trainning data.

	\begin{figure}[H]
		\includegraphics[width=4cm]{figures/1174080/9/teori9.PNG}
            	\centering
           	 \caption{soal 9}
       	 \end{figure}

        \item Berikan contoh dengan ilustrasi penjelasan dari Initial latent vector approximation.

Latent vector approdimation kemampuan untuk membuat gamar yang realistis dan tajam serta menghasilkan gambar wajah pada usia target.

	\begin{figure}[H]
		\includegraphics[width=4cm]{figures/1174080/9/teori2,10.PNG}
            	\centering
           	 \caption{soal 10}
       	 \end{figure}

        \item Berikan contoh perhitungan latent vector optimization.

Perhitungan lantent optimization menggunakan metode yang relatif sederhana, tergantung pada jumlah kecil parameter yang diperlukan, sehingga pada latent optimization dapat memetakan setiap gambar x dari dataset ke vektor acak dimensi rendah zi dalam ruang laten z.

	\begin{figure}[H]
		\includegraphics[width=4cm]{figures/1174080/9/teori11.PNG}
            	\centering
           	 \caption{soal 11}
       	 \end{figure}

\end{enumerate}

\subsection{Praktek}
\begin{enumerate}

	\item Jelaskan bagaimana cara ekstrak file dataset Age-cGAN menggunakan google colab.
Menggunakan Google Colab, dimana membuat notebooks baru, kemudian membuat ekstraksi file dari link dataset.

		\lstinputlisting[firstline=1, lastline=4]{src/1174080/9/chapter9.py}

	\item Jelaskan bagaimana kode program bekerja untuk melakukan load terhadap dataset yang sudah di ekstrak, termasuk bagaimana penjelasan kode program perhitungan usia.
Dibawah ini merupakan code untuk melakukan fungsi perhitungan usia.

		\lstinputlisting[firstline=6, lastline=31]{src/1174080/9/chapter9.py}

	\item Jelaskan bagaimana kode program The Encoder Network bekerja dijelaskan dengan bahawa awam dengan ilustrasi sederhana.
Proses Encoder berfungsi untuk mempelajari pemetaan terbalik dari gambar wajah dan kondisi usia dengan vector latent Z.

		\lstinputlisting[firstline=33, lastline=73]{src/1174080/9/chapter9.py}

	\item Jelaskan bagaimana kode program The Generator Network bekerja  dengan ilustrasi sederhana.
Proses Generator agar bekerja dengan baik dibutuhkan representasi dari gambar wajah dan vector kondisi sebagai inputan yang menghasilkan sebuah gambar.

		\lstinputlisting[firstline=75, lastline=104]{src/1174080/9/chapter9.py}

        	\item Jelaskan bagaimana kode program The Discriminator Network bekerja dijelaskan dengan bahawa awam dengan ilustrasi sederhana.
Proses Discriminator untuk membedakan antara gambar asli dan gambar palsu.

		\lstinputlisting[firstline=116, lastline=148]{src/1174080/9/chapter9.py}

        	\item Jelaskan bagaimana kode program Training cGAN bekerja dijelaskan dengan bahawa awam dengan ilustrasi sederhana.
Proses Training cGAN ini dengan load file .mat pada dataset lalu epoch sebanuak 500 kali.

		\lstinputlisting[firstline=150, lastline=167]{src/1174080/9/chapter9.py}

        	\item Jelaskan bagaimana kode program Initial dan latent vector approximation bekerja dijelaskan dengan bahawa awam dengan ilustrasi sederhana.
Initial dan Latent Vector Approximation bekerja melakukan predicsi epoch yang telah di buat sebanyak 500 kali, dan nanti hasilnya ada di folder result.

		\lstinputlisting[firstline=169, lastline=217]{src/1174080/9/chapter9.py}


\end{enumerate}

\subsection{Penanganan Error}

\subsection{Bukti Tidak Plagiat}
\begin{figure}[H]
\centering
	\includegraphics[width=4cm]{figures/1174080/9/plagiat9.PNG}
	\caption{Bukti Tidak Melakukan Plagiat Chapter 9}
\end{figure}